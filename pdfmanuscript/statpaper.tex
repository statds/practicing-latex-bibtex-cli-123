\documentclass[12pt]{article}

\usepackage{amsmath}
\usepackage[margin = 1in]{geometry}
\usepackage{graphicx}
\usepackage{booktabs}
\usepackage{natbib}

\usepackage{lipsum}
\usepackage[colorlinks=true, citecolor=blue]{hyperref}




\title{Stats Paper}
\author{Carol Li\\
    STAT 3494W\\
    University of Connecticut
}

\begin{document}
\maketitle{Assignment 2}

\begin{abstract}
  Linear regression is a vital statistical technique for modeling relationships between variables. 
  This abstract offers a concise overview of its core concepts and applications, highlighting its importance 
  in making predictions and drawing insights from data.
\end{abstract}


\section{Introduction}
\label{sec:intro}

Use this section to answer three questions:
Why is the topic important/interesting?
What has been done on this topic in the literature?
What is your contribution?

\lipsum[1]


The rest of the paper is organized as follows.
The linear model equation are presented in Section~\ref{sec:eq}.
The linear regression equation are presented in Section~\ref{sec:eq2}.
The table is shown in Section~\ref{sec:tab}.
A figure is shown in Section~\ref{sec:fig}.


\section{Linear Model Equation}
\label{sec:eq}

Linear model
\begin{equation}
  \label{slope}
  y = mx + b
\end{equation}

Equation~\eqref{eq:slope} is interesting. $m$ means slope.  
\lipsum{1}

\section{Linear Regression Equation}
\label{sec:eq2}

Linear regression equation
\begin{equation}
  \label{eq:lm}
  y = beta_0 + beta_1*x + E,
\end{equation}
Equation~\eqref{eq:lm} is interesting. $E$ means error term.  

\lipsum[2]



\section{Table} 
\label{sec:tab}

Table~\ref{tab:table1} summarizes some example draws from some distributions.
\lipsum[3]


\begin{table}[h!]
  \begin{center}
    \caption{first table}
    \label{tab:table1}
    \begin{tabular}{l|c|r} 
      \textbf{Value 1} & \textbf{Value 2} & \textbf{Value 3}\\
      $\alpha$ & $\beta$ & $\gamma$ \\
      \hline
      1 & 1111.1 & a\\
      2 & 12.3 & b\\
      3 & 0.999999 & c\\
    \end{tabular}
  \end{center}
\end{table}





\section{Figure}
\label{sec:fig}

Figure~\ref{fig:plot} shows the distance against the speed from this dataset.


\begin{figure}[tbp]
  \centering
  \includegraphics[width=\textwidth]{plot.pdf}
  \caption{first figure.}\label{fig:plot}
\end{figure}

\lipsum[2]

\section{Citations}
\label{sec:cite}

\citet{ananth1997regression} did something 
\lipsum[1]

A lot of work has been done \citep[e.g.,][]{ananth1997regression}.
\lipsum[2]
Some parametric bootstrap sample size approach was proposed by
\citet{hu2023hierarchical}. 
\citep{poole1971assumptions}.

\bibliography{refs}
\bibliographystyle{mcap}

\end{document}